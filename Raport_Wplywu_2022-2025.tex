\documentclass[a4paper,11pt]{article}
\usepackage[utf8]{inputenc}
\usepackage[T1]{fontenc}
\usepackage[polish]{babel}
\usepackage{graphicx}
\usepackage{hyperref}
\usepackage{booktabs}
\usepackage{geometry}
\usepackage{longtable}
\usepackage{xcolor}
\usepackage{float}
\usepackage{array}

\geometry{margin=2.5cm}

\hypersetup{
    colorlinks=true,
    linkcolor=blue,
    filecolor=magenta,      
    urlcolor=blue,
}

\title{\textbf{Raport Wpływu: Użyteczność Kliniczna Zidentyfikowanych Korelacji Genotypowo-Fenotypowych (2022-2025)}}
\author{Tomasz Gambin et al.}
\date{\today}

\begin{document}

\maketitle

\begin{abstract}
Niniejszy raport przedstawia kliniczny wpływ korelacji genotypowo-fenotypowych zidentyfikowanych przez Tomasza Gambina i współpracowników. W okresie ewaluacji 2022-2025 korelacje te były szeroko wykorzystywane w diagnostyce klinicznej na całym świecie, przyczyniając się bezpośrednio do diagnozy setek pacjentów.
\end{abstract}

\section{Metodologia}
Dane do niniejszego raportu zostały pobrane z bazy \href{https://www.ncbi.nlm.nih.gov/clinvar/}{ClinVar} (National Center for Biotechnology Information) przy użyciu narzędzi Entrez E-utilities. Proces automatycznego pobierania i filtrowania danych obejmował następujące kroki:

\begin{enumerate}
    \item \textbf{Pobieranie danych}: Dla zdefiniowanej listy genów wyszukano zgłoszenia (submissions) spełniające kryteria:
    \begin{itemize}
        \item Status kliniczny: \textit{Pathogenic} lub \textit{Likely Pathogenic}.
        \item Data zgłoszenia: 1 stycznia 2022 -- obecnie (2025).
    \end{itemize}
    \item \textbf{Filtrowanie}: Z surowych danych wykluczono:
    \begin{itemize}
        \item Duże zmiany strukturalne (CNV) o wielkości powyżej 500 kpz (500,000 par zasad), aby wyeliminować duże delecje/duplikacje obejmujące wiele genów (np. zespoły mikrodelecji), które nie są specyficznym wynikiem dla analizowanego genu (np. delecje 22q11.21 dla \textit{TANGO2}).
        \item Zgłoszenia, których opis fenotypu sugerował znane zespoły mikrodelecji (np. "22q11.2 deletion syndrome").
    \end{itemize}
    \item \textbf{Analiza}: Pozostałe zgłoszenia zostały zliczone i przypisane do odpowiednich genów oraz ośrodków zgłaszających.
\end{enumerate}

\section{Dlaczego odkrycia były możliwe - wkład PW/TG}
Odkrycia opisane w niniejszym raporcie były możliwe dzięki zastosowaniu nowatorskich metod obliczeniowych i algorytmicznych, opracowanych lub współtworzonych przez Tomasza Gambina. Kluczowe elementy to:

\begin{itemize}
    \item \textbf{Potoki do analizy wariantów i integracja danych}: Wdrożenie zintegrowanych systemów analizy danych NGS (WES/WGS) pozwalających na efektywne łączenie informacji o wariantach SNV i CNV.
    \begin{itemize}
        \item \href{https://pubmed.ncbi.nlm.nih.gov/26195989/}{Gambin et al. (2015) - Secondary findings and carrier test frequencies...}
    \end{itemize}
    \item \textbf{Reanalizy danych}: Systematyczne reanalizy "negatywnych" przypadków, które doprowadziły do nowych odkryć, np. w pracach:
    \begin{itemize}
        \item \href{https://pubmed.ncbi.nlm.nih.gov/28934986/}{Gambin et al. (2017) - Identification of novel candidate disease genes...}
        \item \href{https://pubmed.ncbi.nlm.nih.gov/25259927/}{Yamamoto et al. (2014) - ANKLE2}
    \end{itemize}
    \item \textbf{Narzędzia do detekcji CNV (HMZDelFinder)}: Autorskie narzędzie pozwalające na detekcję homo/hemi-zygotycznych CNV w danych eksomowych, co było kluczowe np. dla genu \textit{TANGO2}.
    \begin{itemize}
        \item \href{https://pubmed.ncbi.nlm.nih.gov/27980096/}{Gambin et al. (2017) - Homozygous and hemizygous CNV detection...}
    \end{itemize}
    \item \textbf{Konstrukcje mikromacierzy}: Projektowanie dedykowanych mikromacierzy (w tym z pokryciem eksonowym, np. V8), które do 2017 roku zostały użyte w badaniu ponad 46,000 przypadków (kluczowe np. dla \textit{FOXF1}).
\end{itemize}

\section{Szczegółowy opis genów i publikacji}

\subsection{Nowe Geny}
Poniższa tabela przedstawia listę nowych genów chorobowych zidentyfikowanych przy udziale Tomasza Gambina. Są to geny, dla których po raz pierwszy opisano związek z chorobą u ludzi.

\begin{longtable}{|>{\itshape}l|l|p{7cm}|c|}
\hline
\large\textbf{Gen} & \large\textbf{OMIM} & \large\textbf{Publikacje} & \large\textbf{Afiliacja PW} \\
\hline
\endhead
TANGO2 & \href{https://omim.org/entry/616878}{616878} & \href{https://pmc.ncbi.nlm.nih.gov/articles/PMC4746334/}{Lalani et al. (2016)} \newline \href{https://pubmed.ncbi.nlm.nih.gov/28934986/}{Gambin et al. (2017)} & Tak \\
\hline
PSMD12 & \href{https://omim.org/entry/617516}{617516} & \href{https://www.cell.com/ajhg/fulltext/S0002-9297(17)30003-4}{Küry et al. (2017)} \newline \href{https://pubmed.ncbi.nlm.nih.gov/28934986/}{Gambin et al. (2017)} & Tak \\
\hline
TRIP12 & \href{https://omim.org/entry/604506}{604506} & \href{https://pmc.ncbi.nlm.nih.gov/articles/PMC5543723/}{Zhang et al. (2017) - TG first-coauthor} \newline \href{https://pubmed.ncbi.nlm.nih.gov/28934986/}{Gambin et al. (2017)} & Tak \\
\hline
ANKLE2 & \href{https://omim.org/entry/616062}{616062} & \href{https://pubmed.ncbi.nlm.nih.gov/25259927/}{Yamamoto et al. (2014)} & Tak \\
\hline
TUBGCP2 & \href{https://omim.org/entry/617817}{617817} & \href{https://www.cell.com/ajhg/fulltext/S0002-9297(19)30357-X}{Mitani et al. (2019)} & Tak \\
\hline
COPA & \href{https://omim.org/entry/616414}{616414} & \href{https://pmc.ncbi.nlm.nih.gov/articles/PMC4513663/}{Watkin et al. (2015)} & - \\
\hline
DVL1 & \href{https://omim.org/entry/616331}{616331} & \href{https://pmc.ncbi.nlm.nih.gov/articles/PMC4385180/}{White et al. (2015)} & Tak \\
\hline
SOHLH1 & \href{https://omim.org/entry/610224}{610224} & \href{https://academic.oup.com/jcem/article-abstract/100/5/E808/2829757}{Bayram et al. (2015)} & - \\
\hline
MIPEP & \href{https://omim.org/entry/602241}{602241} & \href{https://link.springer.com/article/10.1186/s13073-016-0360-6}{Eldomery et al. (2016)} & - \\
\hline
PRUNE1 & \href{https://omim.org/entry/617413}{617413} & \href{https://www.cell.com/neuron/fulltext/S0896-6273(15)00837-5}{Karaca et al. (2015)} & - \\
\hline
VARS1 & \href{https://omim.org/entry/617802}{617802} & \href{https://www.cell.com/neuron/fulltext/S0896-6273(15)00837-5}{Karaca et al. (2015)} & - \\
\hline
DHX37 & \href{https://omim.org/entry/617362}{617362} & \href{https://www.cell.com/neuron/fulltext/S0896-6273(15)00837-5}{Karaca et al. (2015)} & - \\
\hline
RDH11 & \href{https://omim.org/entry/607849}{607849} & \href{https://pmc.ncbi.nlm.nih.gov/articles/PMC4189905/}{Xie et al. (2014)} & - \\
\hline
\end{longtable}
\subsection{Poszerzenie Fenotypu}
W tej sekcji przedstawiono geny, dla których badania Tomasza Gambina przyczyniły się do istotnego poszerzenia spektrum fenotypowego lub lepszego zrozumienia mechanizmu choroby.

\begin{longtable}{|>{\itshape}l|l|p{7cm}|c|}
\hline
\large\textbf{Gen} & \large\textbf{OMIM} & \large\textbf{Publikacje} & \large\textbf{Afiliacja PW} \\
\hline
\endhead
ACTG2 & \href{https://omim.org/entry/619431}{619431} & \href{https://journals.plos.org/plosgenetics/article?id=10.1371/journal.pgen.1004258}{Wangler et al. (2014)} & Tak \\
\hline
PGM3 & \href{https://omim.org/entry/172100}{172100} & \href{https://pmc.ncbi.nlm.nih.gov/articles/PMC4085583/}{Stray-Pedersen et al. (2014)} & - \\
\hline
CORO1A & \href{https://omim.org/entry/605000}{605000} & \href{https://pmc.ncbi.nlm.nih.gov/articles/PMC4386834/}{Stray-Pedersen et al. (2014)} & - \\
\hline
\end{longtable}
\subsection{Rozwój płuc i sekwencje niekodujące}
Poniższa tabela zawiera geny kluczowe dla rozwoju płuc, w przypadku których badania koncentrowały się na roli sekwencji regulatorowych i niekodujących (CNV, SNV w regionach niekodujących).

\begin{longtable}{|>{\itshape}l|l|p{7cm}|c|}
\hline
\large\textbf{Gen} & \large\textbf{OMIM} & \large\textbf{Publikacje} & \large\textbf{Afiliacja PW} \\
\hline
\endhead
FOXF1 & \href{https://omim.org/entry/601089}{601089} & \href{https://pmc.ncbi.nlm.nih.gov/articles/PMC8284783/}{Szafranski et al. (2021)} & Tak \\
\hline
TBX4 & \href{https://omim.org/entry/601719}{601719} & \href{https://pmc.ncbi.nlm.nih.gov/articles/PMC6369446/}{Karolak et al. (2019)} & Tak \\
\hline
FGF10 & \href{https://omim.org/entry/602115}{602115} & \href{https://pmc.ncbi.nlm.nih.gov/articles/PMC6369446/}{Karolak et al. (2019)} & Tak \\
\hline
\end{longtable}

\section{Dowody Wizualne}
Poniższe wykresy ilustrują skalę i zasięg oddziaływania klinicznego w latach 2022-2025.

\begin{figure}[H]
    \centering
    \includegraphics[width=0.9\textwidth]{images/impact_timeline_pl.png}
    \caption{Liczba zgłoszeń wariantów sklasyfikowanych jako Pathogenic lub Likely Pathogenic w latach 2022-2025}
\end{figure}

\begin{figure}[H]
    \centering
    \includegraphics[width=0.9\textwidth]{images/impact_by_gene_pl.png}
    \caption{Liczba zgłoszeń wariantów sklasyfikowanych jako Pathogenic lub Likely Pathogenic wg genu}
\end{figure}

\begin{figure}[H]
    \centering
    \includegraphics[width=0.9\textwidth]{images/impact_map_pl.png}
    \caption{Globalny zasięg ośrodków diagnostycznych}
\end{figure}

\section{Statystyki (2022-2025)}
Tabela przedstawia liczbę wariantów sklasyfikowanych jako Pathogenic lub Likely Pathogenic w bazie ClinVar w okresie 2022-2025.

\begin{longtable}{lr}
\toprule
\large\textbf{Gen} & \large\textbf{Liczba zgłoszeń P/LP (2022-2025)} \\
\midrule
\textit{TANGO2} & 90 \\
\textit{PGM3} & 89 \\
\textit{TRIP12} & 78 \\
\textit{TBX4} & 61 \\
\textit{ACTG2} & 59 \\
\textit{FOXF1} & 36 \\
\textit{PRUNE1} & 29 \\
\textit{VARS1} & 25 \\
\textit{CORO1A} & 24 \\
\textit{MIPEP} & 23 \\
\textit{PSMD12} & 18 \\
\textit{FGF10} & 16 \\
\textit{DHX37} & 16 \\
\textit{COPA} & 13 \\
\textit{ANKLE2} & 12 \\
\textit{DVL1} & 11 \\
\textit{SOHLH1} & 9 \\
\textit{RDH11} & 8 \\
\textit{TUBGCP2} & 7 \\
\midrule
\textbf{SUMA} & \textbf{624} \\
\bottomrule
\end{longtable}

\end{document}
